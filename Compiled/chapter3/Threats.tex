\section{Impact on the Network Ecosystem}\label{sec:threats}



%To understand the implications of these factors, we need to first look at the impact that this growth is having on the various stakeholders of the Internet ecosystem before delving deeper into the promises that \emph{Smart Data Pricing} (SDP) holds \cite{SDPForum}. In the following discussion, we explore the current challenges and threats from the perspective of carriers, consumers, and content providers.

%The Internet ecosystem today suffers from threats to both its \emph{sustainability} and \emph{economic viability}. The sustainability concern is due to the near doubling of demand for mobile data every year  \cite{CiscoVNI}, which brings into question the feasibility of finding any purely technical solutions to this problem. Additionally, changes in the policy to make more spectrum available, although useful, is unlikely to provide any long-term solution.  For example, the US Government's initiative in releasing an additional 50 MHz of spectrum into the market for six carriers to compete for is widely viewed as being quite insufficient \cite{tollfree}. 

%The latter concern regarding the economic viability of Internet access is due to the large overage charges and throttling that ISPs (e.g., AT\&T and Verizon) are now imposing on their customers to penalize demand. To understand the implications of these factors, we need to first look at the impact that this growth is having on the various stakeholders of the Internet ecosystem before delving deeper into the promises that \emph{Smart Data Pricing} (SDP) holds \cite{SDPForum}. In the following discussion, we explore the current challenges and threats from the perspective of carriers, consumers, and content providers.

\subsection{ISPs' Traffic Growth}    

%The Cisco Visual Networking Index \cite{CiscoVNI} predicted in 2012 that mobile data traffic\footnote{``Mobile data traffic'' includes handset-based data traffic, such as text messaging, multimedia messaging, and handset video services, as well as traffic generated by notebook cards and mobile broadband gateways.} will grow at a compound annual growth rate (CAGR) of 78\% between 2011 and 2016, reaching a volume of 10.8 exabytes per month in 2016. By that time, Cisco predicted that the average smartphone user will consume 2.6 GB per month, as opposed to 150 MB/month in 2011 \cite{CiscoVNImobile}. This growth in per-device data consumption is fueled by the demand for bandwidth-intensive apps: for example, Allot Communications reports that usage of Skype's video calling service grew by 87\% in 2010, while usage of Facebook's mobile app grew by 267\% \cite{allot}. Devices themselves are also contributing to the increase in traffic volume; Apple's iPad 3 quadrupled the screen resolution from the iPad 2, allowing videos of higher quality to be streamed to the device \cite{apple}.

%Cisco predicted that by 2016, Wi-Fi and mobile data will comprise 61\% of all Internet traffic, with wired traffic comprising the remaining 39\% \cite{CiscoVNI}. Although the faster growth rate in mobile data traffic is a major concern, growing congestion on wired networks both at the edge and in the networks' middle mile also poses a significant problem. 
%-- consumer traffic from fixed IPs is expected to grow at a CAGR of 28\% per year from 2011-2016, while managed IP traffic is expected to sustain a 21\% CAGR \cite{CiscoVNI}.\footnote{Cisco's definition of ``consumer'' includes fixed IP traffic generated by households, university populations, and Internet cafes, `while ``managed IP'' includes corporate IP WAN traffic and IP transport of TV and VoD.}) 

By 2016, ISPs are expected to carry 18.1 petabytes per month in managed IP traffic.\footnote{Cisco's definition of ``managed IP'' includes traffic from both corporate IP wide area networks and IP transport of television and video-on-demand.} But this growth is causing concern among ISPs, as seen during Comcast's initiative to cap their wired network users to 300 GB per month \cite{Comcast-cap}. Even back in 2008, Comcast made headlines with their decision (since reversed) to throttle Netflix as a way to curb network congestion \cite{Comcast-Level3}. Video streaming from services like Netflix, Youtube, and Hulu, are a major contributor to wired network traffic. In fact Cisco predicts that by 2016 fixed IPs will generate 40.5 petabytes of Internet video per month \cite{CiscoVNI}.
  
Rural local exchange carriers (RLECs) are also facing congestion in their wired networks due to the persistence of the middle-mile problem for RLECs. Although the cost of middle mile bandwidth has declined over the years (because of an increase in the DSL demand needed to fill the middle mile), the bandwidth requirements of home users have increased quite sharply \cite{vglass-2}. Still, the average speed provided to rural customers today fails to meet the Federal Communications Commission's (FCC) broadband target rate of 4 Mbps downstream speed for home users. The cost of middle mile upgrades to meet this target speed will be substantial and is a barrier to digital expansion in the rural areas \cite{vglass-2}. Research on access pricing as a mechanism to bring down middle mile investment costs by reducing the RLEC's peak capacity and over-provisioning needs can therefore also help in bridging the digital divide.

\subsection{Consumers' Cost Increase}

Network operators have begun to pass some of their network costs to consumers through various penalty mechanisms (e.g., overage fees) and increasing the cost of Internet subscriptions. For instance, when Verizon announced in July 2012 that they were offering shared data plans for all new consumers and discontinuing their old plans, many consumers ended up with higher monthly bills \cite{nyt-shared}. To remain within monthly data caps, consumers are increasingly relying on usage-tracking and data compression apps (e.g., Onavo, WatchDogPro, DataWiz) \cite{datawiz} that help to avoid overage fees. Such trends are common in many parts of the world; in South Africa, for instance, consumers use ISP-provided usage-tracking tools \cite{chetty2012you} to stay within the data caps. Similarly in the U.S., research on in-home Internet usage has shown that many users are concerned about their wired Internet bills and would welcome applications for tracking their data usage and controlling bandwidth rates on in-home wired networks \cite{chetty2011my,netlimiter}. Empowering users to monitor their data usage and control their spending has led to a new area of research that considers economic incentives and human-computer interaction (HCI) aspects in a holistic manner \cite{sigchi}. 

\subsection{Application Developers' Perspective}

Introducing pricing schemes that create a feedback-control loop between the client side device and network backend devices requires new mobile applications that will support such functionalities. However, most mobile platforms in use today (e.g., iOS, Android, and Windows) have different levels of platform openness.  The iOS platform for iPhones and iPads has several restrictions: it strictly specifies what kind of applications can run in the background and further prevents any access other than the standard application programming interfaces (APIs). For example, obtaining an individual application's usage and running a pricing app in the background are prohibited. By contrast, the Android and Windows platforms allow these features, e.g., introducing an API to report individual applications' usage to third-party apps.

An interesting direction to overcome these limitations is to initiate the creation of open APIs between user devices and an ISP's billing systems. For example, this can allow the user devices connected to the ISP's network to easily fetch current pricing, billing, and usage information from the network operator, while also allowing the ISP to easily test and deploy new pricing schemes through the standardized interface. Such an API would foster innovations in pricing for both consumers and providers.

Additionally, new pricing plans create an opportunity for developers to optimize their app according to changing pricing conditions. For instance, some apps that require preloading content, such as magazine apps, might time these preloading downloads so as to coincide with lower-price times, thus saving users money \cite{sigchi}. This sensitivity to price might even improve users' experience, as lower prices generally occur during times of lower congestion and higher throughput. Shifting usage so as to save money could be especially significant for video apps, as these tend to have higher usage volumes.\footnote{Some video apps cannot shift usage due to legal restrictions on caching content. However, many apps like YouTube own the rights to their video content.} Such adaptation would also require an API allowing apps to access the network prices in real time.

% Wireless ISPs' current billing systems (including 2G, 3G, and 4G) heavily depend on the RADIUS (Remote Authentication Dial In User Service) protocol, which supports centralized Authentication, Authorization, and Accounting (AAA) for users or devices to use a network service \cite{rfc2865}. In particular, RADIUS accounting \cite{rfc2886} is well suited to support usage-based pricing, since it can keep track of the usage of individual sessions belonging to each user. Interim-update messages to each session can be sent periodically to update the usage information. However, RADIUS accounting lacks support for dynamic pricing plans, which require time-of-day usage at various time scales\footnote{Note that interim update messages are sent periodically when a session joins the system, and hence, the time interval for interim updates should be kept low to support sending time-of-day usage, which may introduce significant control overhead.} Therefore, extending these protocols to support new pricing mechanisms, standardizing interfaces, and the creation of open APIs between network operators and application developers will be interesting directions for future research in this area.
  
%Additionally, empowering users to monitor and control of their data usage requires understanding of the intersection between the economic incentives and human-computer interaction (HCI) aspects \cite{sigchi}. 

\subsection{Software/Hardware Limitations}

Wireless ISPs' current billing systems (including 2G, 3G, and 4G) heavily depend on the RADIUS (Remote Authentication Dial In User Service) protocol, which supports centralized Authentication, Authorization, and Accounting (AAA) for users or devices to use a network service \cite{rfc2865}. In particular, RADIUS accounting \cite{rfc2886} is well suited to support usage-based pricing, since it can keep track of the usage of individual sessions belonging to each user. Yet individual session lengths are often quite long, making it difficult to retrieve usage at the smaller timescales needed for dynamic pricing.

RADIUS account sessions are initiated by the Network Access Server (NAS) when a user first attempts to connect to the network: the NAS sends a user's login credentials to the RADIUS server, which compares the credentials to a secure database. The RADIUS server then \emph{authenticates} the session and \emph{authorizes} it to access different functionalities on the network. Once this session has been initiated, a start record is created in the RADIUS logs. Interim \emph{accounting} request messages can be sent periodically to update the usage information. When the end user terminates the connection, the NAS sends a stop message to the RADIUS server and a stop record is created that stores the total usage volume of that session. Since these RADIUS sessions can have very long durations, up to several hours, RADIUS logs cannot be used to calculate usage at smaller timescales.\footnote{Note that interim update messages are sent periodically when a session joins the system, and hence, the time interval for interim updates should be kept low to support sending time-of-day usage, which may introduce significant control overhead.} Moreover, the RADIUS log has no information on the type of application(s) corresponding to each session. While one session may encompass usage from multiple apps used in parallel, in some cases individual apps initiate new sessions; thus, the concept of a ``session'' cannot be definitively tied to an individual app \cite{rfc2886}.

Changing the RADIUS protocol to deliver usage estimates at a smaller time granularity would require significant overhead in both control signaling and storing RADIUS records. A perhaps easier alternative would be to record usage at the network edge, i.e., on client devices--such functionality already exists, but this approach would be vulnerable to users' deliberately falsifying the usage recorded on their device. Similarly, RADIUS logs do not contain any information on per-application usage, but client devices can easily obtain this information. Thus, application-specific pricing could also benefit from usage tracking functionalities on the end user devices. Some verification procedures could be implemented to guard against user tampering, e.g., comparing the total monthly usage measured by RADIUS servers and client devices, but would require careful design and might not be completely secure.

% Another interesting direction is the creation of an open API between user devices and an ISP's billing systems. The open API will foster innovations in pricing for both consumers and providers. For example, the user devices connected to the ISP's network can easily fetch their pricing, billing, and usage information from the network, and the ISP can also easily test and deploy new pricing schemes through the standard interface.

%\emph{Reducing App Bandwidth:} In light of the growing demand for mobile data, a natural question is whether more efficient app development can reduce apps' traffic volume, and more fundamentally whether bandwidth consumption is a consideration during app development. A small online survey of mobile app developers revealed divided opinions \cite{sigchi}. For example, one iOS developer with 4 years of experience and 12 apps told us �No, most app developers do what must be done. Only experienced app developers will look for ways to reduce bandwidth usage� (D1). Developer D2 with 5 years of experience echoed this view: �User experience and interface seem to be the main need in the App Store.� But another iOS developer with 2 years of experience and 3 apps opined that apps do consider data usage: �Bandwidth overhead is very important to the user and should thus be important to the developer too� (D3). However, he added that the �previous version of Facebook app was - because of all UIWebViews - very slow on mobile connections. Reducing the bandwith by using native UI interfaces gave a huge speed bump.� When asked whether TDP plans will hurt app development, developer D1 felt that developers will adapt to new plans, while D3 felt that it won�t have any effect: �I don't think that developers will change their design just because of changes in data plans.�
%
%The developers also provided examples of apps that can actually benefit from TDP: �applications that can cache websites and newsfeeds to read later� (D3), �Dropbox or any file sharing and backup app� (D2), and apps like YouTube that can choose �different resolutions depending on network availability� (D1). But, developer D3 opined: �The Internet�s greatest advantages above all other media is, that it's always up-to-date. It is the fastest media and adding such plans will cut off this advantage. The users want to be up-to-date.� 


\subsection{Content Delivery Issues}

Any change in access pricing has to be studied in the larger context of Internet's net-neutrality and openness. These discussions center around the issues of (a) who should pay the price of congestion (i.e., content providers or consumers) and (b) how such pricing schemes should be implemented (i.e., time-of-day, app-based bundles, etc.).  The major concern with policy change is the possibility of paid prioritization of certain content providers' traffic, price discrimination across consumers, and promoting anti-competitive behavior in bundled offerings of access plus content.  While such developments can indeed hurt the network ecosystem, one aspect that should receive more attention is the threat to data usage even under simple usage-based or tiered data plans.  As Internet users become more cautious about their data consumption \cite{BostonGlobe}, content providers are providing new options to downgrade the quality of experience (QoE) for their users to help them save money.  For instance, Netflix has started allowing ``\emph{users to dial down the quality of streaming videos to avoid hitting bandwidth caps}''  \cite{Newman}. Additionally, it is ``\emph{giving its iPhone customers the option of turning off cellular access to Netflix completely and instead relying on old-fashioned Wi-Fi to deliver their movies and TV shows}'' \cite{Fitchard}.  Thus, the ecosystem today is being driven by an attitude of penalizing demand and lessening consumption through content quality degradation. 

Network researchers are investigating these issues broadly along two lines of work: (i) opportunistic content caching, forwarding, and scheduling, and (ii) budget-aware online video adaptation. Opportunistic content delivery involves the smart utilization of unused resources to deliver higher QoE; for example, to alleviate the high cost of bulk data transfers, Marcon et al. \cite{marcon2012netex} proposed utilizing excess bandwidth (e.g., at times of low network traffic) to transmit low-priority data. Since this data transmission does not require additional investment from ISPs, they can offer this service at a discount, relieving data transfer costs for clients. While utilizing excess bandwidth introduces some technical issues (e.g., the potential for resource fluctuations), a prototype implementation has shown that they are not insurmountable \cite{laoutaris2011inter}. The second stream of works on online video adaptation systems, such as Quota Aware Video Adaptation (QAVA) \cite{Jiasi}, have focused on sustaining a user's QoE over time by predicting her usage behavior and leveraging the compressibility of videos to keep the user within the available data quota or her monthly budget.  The basic idea here is that the video quality can be degraded by non-noticeable amounts from the beginning of a billing cycle based on the user's predicted usage so as to avoid a sudden drop in QoE due to throttling or overage penalties when the monthly quota is exceeded. This relates to the SDP theme of enabling self-censorship of usage and QoE on the client side device through user-specified choices. %A common feature of these new research directions is that they address the concerns of consumers and content providers by accounting for both technological and economic factors.

\subsection{Regulatory Concerns}

Pricing in data networks has remained a politically charged issue, particularly for pricing mechanisms that could potentially create incentives for price discrimination, non-neutrality, and other anti-competitive behavior through app-based pricing or bundling of access and content.  Academics have already cautioned that the ongoing debate on network neutrality in the U.S. often overlooks service providers' need for flexibility in exploring different pricing regimes \cite{Yoo}:   
\begin{quote}
\emph{Restricting network providers' ability to experiment with different protocols may also reduce innovation by foreclosing applications and content that depend on a different network architecture and by dampening the price signals needed to stimulate investment in new applications and content.}
\end{quote}

But faced with the growing problem of network congestion, there has been a monumental shift in the regulatory perspective in the US and other parts of the world.  This sentiment was highlighted in FCC Chairman J. Genachowski's 1 December 2010 statement \cite{FCC}, which recognizes \emph{``the importance of business innovation to promote network investment and efficient use of networks, including measures to match price to cost."}  

