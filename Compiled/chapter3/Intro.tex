\section{Introduction}\label{intro}

Advances in Internet technologies have resulted in an unprecedented growth in demand for data. In particular, demand in the mobile Internet sector is doubling every year \cite{CiscoVNI}. Given the limited wireless spectrum availability, the rate of growth in the supply of wireless capacity (per dollar of investment) is unlikely to match the rate of growth in demand in the long run. Internet Service Providers (ISPs) are therefore turning to new pricing and penalty schemes in an effort to manage the demand on their network, while also matching their prices to cost. But changes in pricing and accounting mechanisms, if not done carefully, can have significant consequences for the entire network ecosystem. Multiple stakeholders in this ecosystem, including operators, consumers, regulators, content providers, hardware and software developers, and architects of network technologies, have all been tackling these issues of charging and allocating limited network resources. Even back in 1974, while writing about the future challenges of computer communication networks, Leonard Kleinrock \cite{Kleinrock} noted: 
\begin{quote}
\emph{[H]ow does one introduce an equitable charging and accounting scheme in such a mixed network system? In fact, the general question of accounting, privacy, security and resource control and allocation are really unsolved questions which require a sophisticated set of tools.}
\end{quote}
While much progress has been made on developing technical solutions, methods, and tools to address these issues, continued growth of the network ecosystem requires developing a better understanding of the underlying economic and policy perspectives. The broader area of \emph{network economics}, which deals with the interplay between technological and economic factors of networks, is therefore receiving more attention from engineers and researchers today. Economic factors like pricing, costs, incentive mechanisms and externalities\footnote{Network externality is the notion that the cost or value of being a part of a network for an individual user depends on the number of other users using that network. For example, the value of a network grows as more users adopt and positive externalities are realized from being able to communicate with other users on the network. Similarly, when many users start to content for limited resources of a bottleneck link of a network, negative externalities from congestion diminish a user's utility from accessing the network.} affect the  adoption outcomes (i.e., success or failure of network technologies) and stability \cite{Sen2010,Joseph2007,wifi-3g-infocom}, influence network design choices \cite{Sen2011-MVF,Sen2009-SVS}, and impact service innovation \cite{Zhang2010}. Conversely, technological limitations and regulatory constraints determine which kind of economic models are most suited to analyze a particular network scenario. This interplay between technology, economics, and regulatory issues is perhaps most easily observed in the case of broadband access pricing, for example, in evaluating the merits of ``flat-rate" versus ``usage-based" pricing or the neutrality of ``volume-based" versus ``app-based" accounting, etc. In this chapter we discuss the current trends in access pricing among service operators, factors that affect these decisions, analytical models and related considerations. In particular, we observe that \emph{Smart Data Pricing}\footnote{SDP is the broad set of ideas and principles that goes beyond the traditional flat-rate or byte-counting models and instead considers pricing as a network management solution. See http://www.smartdatapricing.org.} (SDP) is likely to emerge as an effective way to cope with increased network congestion. These smarter ways to count and treat data traffic illustrate three shifts in the principles of network management:

\begin{enumerate}
\item {\bf \emph{Pricing for end-user Quality of experience (QoE) and not just byte-counting}}: Simple policies like usage-based pricing (byte-counting) (a) force users to pay the same amount per unit of bandwidth consumed irrespective of the congestion levels on the network,\footnote{In 1997, David Clark wrote \cite{DDClark} that ``The fundamental problem with simple usage fees is that they impose usage costs on users regardless of whether the network is congested or not."} and (b) fail to account for the fact that different applications have different bandwidth requirements to attain a certain QoE for the user. SDP should try to match the cost of delivering application-specific desired QoE requirements of the user to the ISP's congestion cost at the time of delivery.     
\item {\bf \emph{Application layer control to impact physical layer resource management}}:  Today's smart devices with their easy to use graphical user interfaces can potentially enable consumer-specified choice for access quality. Whether done manually or in an automated mode, users' specifications of their willingness to pay for their desired QoE of different applications can be taken in as inputs at the APP layer and used to control PHY layer resource allocation and media selection (e.g., WiFi offloading versus 3G). But enabling this requires consumer trials to understand how to design incentives and create interfaces that can be effective in modifying end-user behavior.
\item {\bf \emph{Incorporating edge devices as a part of network management system}}: Instead of managing traffic only in the network core, SDP explores ways to make edge devices (e.g., smart mobile devices and customer-premise equipments like gateways) a part of the network resource allocation and management system. For example, instead of throttling traffic in the network core using the policy charging and rules function (PCRF), the edge devices (e.g., home gateways) themselves could locally regulate demand based on a user's budget, QoE requirements, and network load or available prices. Such measures to push control from the network core out to the end-users, while preserving the end-to-end principles of the Internet, have been gaining attention among networking research groups \cite{M3I}. 
\end{enumerate}

But before delving deeper into pricing ideas, let us pause to address some common misconceptions often encountered in public discourse. First, many believe that the Internet's development cost was borne by the United States Government, and hence that taxpayers have already paid for it. In reality, by 1994 the National Science Foundation supported less than 10\% of the Internet and by 1996 huge commercial investments were being made worldwide \cite{MIT}. 

Second, users often do not realize that the Internet is not free \cite{MIT, chetty2011my} and think its cost structure is the same as that of information goods. In contrast to \emph{information goods}, which tend to have zero marginal costs,\footnote{Marginal cost is the change in the total cost that arises when the quantity produced changes by one unit, e.g., the cost of adding one more unit of bandwidth.} Internet operators incur considerable network management operation and billing costs. MacKie-Mason and Varian \cite{MM-Varian} have shown that while the marginal cost of some Internet traffic can be zero because of \emph{statistical multiplexing}, congestion costs can be quite significant. 
In regard to delivery of bits, it is worthwhile to note that there are some important factors at play: 
\begin{enumerate}[(a)]
\item There is a large and growing variance in the QoE requirements of the different types of applications that consumers are using today, and
\item The network operator's cost of delivery per bit for a given QoE level also has significant variance, ranging from essentially zero marginal cost in uncongested times to very high in congested times.
\item There is also a variance in user's willingness to pay for different types of traffic and QoE levels.
\end{enumerate}

So why not match the right pairs? Most SDP ideas aim to do exactly that, i.e., match the operator's cost of delivering bits to the consumer's QoE needs for different application types at the amount they are willing to spend.

Third, there is a popular misconception that network costs are high because billing costs account for 50\% of telephony costs. Although true for running costs, it is only 4-6\% when depreciation of sunk costs is added \cite{M3I}. Another important cost for wireless operators today is the cost of acquiring new spectrum to support the growing bandwidth needs of the customers. However, spectrum is limited and expensive, and even auction-based spectrum reallocation schemes are projected to fall short of the demand for spectrum. 

Fourth, the belief that better technologies like 4G and offloading mechanisms will solve the problems is already being questioned -- ``The reasons are two-fold: The amount of spectrum made available to U.S. wireless companies is limited, but the carriers have also been sluggish in buying up enough backhaul to support their capacity requirements. There is only so much data that can be crammed into wireless spectrum -- and only so much spectrum available to wireless networks. Thanks to rising mobile data demands, a current wireless spectrum surplus of 225 MHz will become a deficit of 275 MHz by 2014, according to the FCC \cite{CNN-4G}."

%Higher access speeds lead to both higher demand and applications that consume higher amount of bandwidth. 

Fifth, users fear that changes in pricing policy will increase their access fees. This need not always be the case, as one can design incentive mechanisms that reward good behavior (e.g., price discounts in off-peak hours to incentivize shifting of usage demand from peak times). In other words, smarter pricing mechanisms can \emph{increase consumer choices} by empowering users to take better control of how they spend their monthly budget. For example, under time-dependent usage-based pricing \cite{comm-mag,ha2012tube}, users have better control over their monthly bills by choosing not only \emph{how much} they want to consume, but also \emph{when} they do so). Smart data pricing also has to be smart in its implementation and in its user interface design, with careful study of user psychology and human-computer interaction aspects, as we will illustrate in later sections and case studies. 

Lastly, we also need to remember that pricing is related to the market competition and user population density. For the interested reader, an overview of access fees in different parts of the world is provided in Section \ref{Dataplans}.

The following questions provide a useful way to think about SDP: 
\begin{enumerate}[(I)]
\item Why do we need SDP? Isn't network pricing an untouchable legacy?\\
Section \ref{sec:factors} provides an overview of the driving factors behind network congestion, and the challenges that it poses to various stakeholders of the network ecosystem are discussed in Section \ref{sec:threats}. We also discuss the rapid evolution in pricing among network operators and highlight in Section \ref{SDP} how Smart Data Pricing ideas will be useful in finding solutions that can work in today's networks.
\item Haven't other fields already used pricing innovations? What are the key SDP ideas relevant to communication networks?\\
We provide an overview of Internet pricing ideas in the existing literature in Section \ref{sec:congestion}, including some pricing plans from the electricity and transportation industries that can be applied to broadband pricing. Section \ref{sec:econ} provides an overview of a few examples and analytical models of known pricing mechanisms to illustrate key economic concepts relevant to the SDP literature. We also highlight many crucial differences between SDP in communication or data networks and pricing innovations in other industries.
\item Isn't SDP too complex to implement in the real world?\\
Section \ref{sec:TDP} provides a case study of a field trial of ``day-ahead time-dependent pricing" and discusses the model, system design, and user interface design considerations for realizing this plan. It serves to demonstrate both the feasibility of creating such SDP plans for real deployment while also pointing out the design issues that should be kept in mind. The discussion highlights the end-to-end nature of an SDP deployment, which requires developing pricing algorithms, understanding user psychology, designing an effective interface for communicating those prices to users, and implementing an effective system to communicate between the users and ISPs.
\item What are the outstanding problems in enabling SDP for the Internet?\\
SDP is an active area of research in the network economics community and a set of 20 questions and future directions are provided in Section \ref{sec:20Q} for researchers and graduate students to explore. Many of these research questions have been discussed at various industry-academia forums and workshops on SDP \cite{SDPForum,SDP2013}.  
\end{enumerate}


%This chapter is organized as follows: We first provide an overview of the driving factors behind network congestion in Section \ref{sec:factors} and discuss in Section \ref{sec:threats}, the various congestion related issues in the network ecosystem. We then discuss Smart Data Pricing principles and ideas in Section \ref{SDP}. An overview of pricing ideas in the existing literature is given in Section \ref{sec:congestion}, including some pricing plans from the electricity and transportation industries that can be applied to broadband pricing. Section \ref{sec:econ} provides an overview of a few examples and analytical models of known pricing mechanisms to illustrate key economic concepts essential for understanding the SDP literature. We then delve into one such pricing plan in Section \ref{sec:TDP}, namely, ``day-ahead time-dependent pricing,'' in order to illustrate the end-to-end nature of an SDP deployment, which requires developing pricing algorithms, designing an effective interface for communicating those prices to users, and implementing an effective system to communicate between an ISP and its users. Finally, in Section \ref{sec:20Q} we point to some (of many) research questions for broadband access pricing that are still open for further research in this area.

