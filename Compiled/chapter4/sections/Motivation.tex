\section{Motivation}

The Internet is a hugely successful man-made artifact that has changed society
fundamentally.  Imagine the effect a prolonged outage of the Internet would
have: (1) Youngsters wouldn't know how to interact with their peers and how to
spend their leisure time as they increasingly rely on social networks, online
games, YouTube, and other online entertainment offerings.  (2) Manufacturing
would hit a roadblock as the communication paths within and between companies
increasingly rely on the Internet. (3) Control of critical infrastructures
would be hampered as it increasingly relies on the Internet for gathering input
data and propagating control information.

In becoming a hugely successful infrastructure, the usage of the Internet and
thus its structure has also undergone continuous changes. Usage has changed
from dominance by email and FTP in the early days, to the World Wide Web (WWW)
from 1995 to 2000, to peer-to-peer applications (P2P) from 2000 to 2007, back
to the WWW since 2007. These changes are driven in part by the Internet users'
interests as well as how content, including user generated content, is made
available.

When considering the current application mix and traffic streams in the
Internet, the latest buzz is that ``Content is King'' just as Bill
Gates~\cite{Gates-Content-King} predicted in his essay from 1996.  Hereby, the
term content has to be seen very broadly and encompasses everything from
commercially prepared content, \eg broadcast and interactive TV, news, and
software, to user-generated content, \eg videos uploaded to YouTube, and photos
uploaded to Flickr, to interactive activities, \eg online games. Or to quote
Bronfman~\cite{Bronfman-remarks}, the head of a major music producer and
distributor: ``What would the Internet be without `content'? It would be a
valueless collection of silent machines with gray screens. It would be the
electronic equivalent of a marine desert---lovely elements, nice colors, no
life.  It would be nothing.''

The idea of content delivery being the fundamental operation around which to
design future Internet architecture for comes as no surprise. In fact, the idea
of Content-Centric Networking (CCN)~\cite{CCN} is guided by this principle.
Change, however, takes time, and when hundreds of million of devices are
involved, change can only be made slowly. Before such a novel and radically
different architecture such as CCN is available or potentially deployable, the
Internet in its current state must cope with the challenge of delivering
ever-increasing amounts of content to Internet users.

Accordingly, it appears that solely providing connectivity to end users is no
longer sufficient for Internet Service Providers (ISPs).  Yet, connectivity is
a crucial ingredient and some authors, \eg Andrew
Odlyzko~\cite{odlyzko2001content} have opined that enabling communication is
the main task of the Internet network infrastructure. In his paper ``Content is
not king'' he claims that ``Content will have a place on the Internet, possibly
a substantial place. However, its place will likely be subordinate to that of
business and personal communication''.

At this point it is crucial to realize that the producers of content are
usually not the operators of today's Internet infrastructure. Nonetheless, both
content producers and network operators depend on each other. In fact, neither
the Internet infrastructure operators nor the content producers can be
successful without the other.  After all, the content producers want to ensure
that their content gets to Internet users with reasonable performance for which
they need to rely on the network infrastructure.  On the other hand, the
network infrastructure providers have to transport the content and manage the
infrastructure to satisfy the demand for content from their subscribers. It is
this symbiosis between the two parties that motivates our work collaboration
between content producers and network operators in delivering content.

\bigskip 

\noindent\textbf{Outline:} We start this chapter with a short introduction in
Section~\ref{sec:introduction}.  Then, in Section~\ref{sec:ISP}, we set the
stage by providing an overview of today's Internet network infrastructure,
discussing how Internet Service Providers (ISPs) perform traffic engineering,
and reviewing the Domain Name System (DNS), an essential component of any
Web-based content-delivery architecture.   Next, we review current trends in
Internet traffic and the application mix as well as traffic dynamics in
Sections~\ref{sec:traffic} and~\ref{sec:traffic-diversity}.

We finish the overview with a brief summary on the background of content
delivery in Section~\ref{sec:overview}. Here, we assume that the reader is
familiar with the basic architecture of the Web. There are excellent text books
on this topic, \eg~\cite{web-book}.  Given that there are several approaches to
content delivery, we provide a general high level description of how different
Content Delivery Infrastructures work.  Since there are also many peer-to-peer
based content delivery systems we provide a short review of the basic P2P
architectures as well. For additional background on P2P we refer the reader to,
\eg~\cite{p2pbook1,p2pbook2}.

An overview of the current content delivery spectrum is presented in
Section~\ref{sec:content-delivery}. Here we discuss various types of Content
Delivery Infrastructures (CDIs) which range from Web-based Content Distribution
Networks (CDNs) to Hybrid CDNs to peer-to-peer (P2P) systems. Furthermore, in
Section~\ref{sec:challenges} we turn to the challenges that each party involved
in Internet content delivery faces separately today.

Finally, we turn to the state of the art of collaboration between networks and
content providers.  We start by outlining the collaboration incentives for each
member of the content delivery landscape in Section~\ref{sec:Incentives}. Next
we review the collaboration schemes that have been discussed in research as
well as at the Internet Engineering Task Force (IETF) in
Section~\ref{sec:Opportunities}. We briefly introduce the well-known approaches
and summarize their key functions. We then pick two collaboration schemes,
namely the P2P Oracle and the Provider-aided Distance Information System
(PaDIS) for a case study. In Section~\ref{sec:oracle} we discuss the P2P Oracle
with regards to its effect on the P2P system as well as on network operations.
Likewise, the second case study discusses the model of the Provider-aided
Distance Information System in Section~\ref{sec:te-cate}, including a large
scale analysis based on real traffic traces.  Section~\ref{sec:future} outlines
a possible future direction for collaboration between content providers and
network operators.  We conclude this part of the chapter in
Section~\ref{sec:conclusion}.


\bigskip 

\noindent\textbf{Summary:} This chapter builds upon the student's
basic knowledge of how the Internet infrastructure operates, i.e., as a network
of networks. After reading this chapter the student should have a fundamental
understanding about how content distribution via the Internet works today, what
the challenges are, and which opportunities lie ahead. Moreover, the chapter
points out how all parties---including end users---can benefit from the
collaboration between ISPs and content providers. Indeed, simple, almost
intuitive, means will enable such collaboration.
