\section{Challenges in Content Delivery}\label{sec:challenges}

The challenges that CDIs and P2P systems are faced with are based on the fact
that they are unaware of the underlying network infrastructure and its
conditions. In the best case, they can try to detect and infer the topology and
state of the ISP's network through measurements, but even with large scale
measurements, it is a difficult task, especially if accuracy is necessary.
Furthermore, when it comes to short-term congestion and/or avoiding network
bottlenecks, measurements are of no use. In the following we describe the
challenges those systems face in more detail.

\subsection{Content Delivery Infrastructures (CDIs)}\label{sec:CDIs-and-Challenges}

From the viewpoint of the end-users and ISPs, the redirection schemes employed
by existing CDIs have three major limitations:

\paragraph{Network Bottlenecks.} Despite the traffic flow optimization
performed by CDIs, the assignment of end-user requests to servers by CDIs may
still result in sub-optimal content delivery performance for the end-users.
This is a consequence of the limited information CDIs have about the network
conditions between the end-user and their servers. Tracking the ever changing
conditions in networks, \ie through active measurements and end-user reports,
incurs an overhead for the CDI without a guarantee of performance improvements
for the end-user.  Without sufficient information about the network paths
between the CDI servers and the end-user, any assignment performed by the CDI
may lead to additional load on existing network bottlenecks, or to the creation
of new bottlenecks.


\paragraph{User Mis-location.} DNS requests received by the CDI DNS servers
originate from the DNS resolver of the end-user, not from the end-user itself.
The assignment is therefore based on the assumption that end-users are close to
their DNS resolvers. Recent studies have shown that in many cases this
assumption does not hold~\cite {Precise:Mao2002,DNS-IMC-2010}. As a result, the
end-user is mis-located and the server assignment is not optimal. As a response
to this issue, DNS extensions have been proposed to include the end-user IP
information~ \cite{DNS-extension-IP-client}.  \paragraph{Content Delivery Cost}
Finally, CDIs strive to minimize the overall cost of delivering huge amounts of
content to end-users. To that end, their assignment strategy is mainly driven
by economic aspects.  While a CDI will always try to assign users in such a way
that the server can deliver reasonable performance, this can again result in
end-users not being directed to the server able to deliver best performance.

\subsection{Peer-to-Peer Networks (P2P)}\label{sec:P2P-and-Challenges}

P2P traffic often starves other applications like Web traffic of
bandwidth~\cite{swxzz-hptp-07}.  This is because most P2P systems rely on
application layer routing based on an overlay topology on top of the Internet,
which is largely independent of the Internet routing and
topology~\cite{abfw-mendg-04}. This can result in a situation where a node in
Frankfurt downloads a large content file from a node in Sydney, while the same
information may be available at a node in Berlin.  As a result P2P systems use
more network resources due to traffic crossing the underlying network multiple
times. For more details and information on P2P systems, see
Section~\ref{sec:P2P}.


\subsection{Internet Service Providers (ISPs)}\label{sec:ISPs-and-Challenges}

ISPs face several challenges regarding the operation of their network
infrastructure. With the emergence of Content, and especially distributed
content delivery, be it from CDIs or P2P networks, these operational challenges
have increased manifold.

\paragraph{Network Provisioning.} Provisioning and operation a network means
running the infrastructure at its highest efficiency. To ensure this, new
cables as well as the peering points with other networks need to be established
and/or upgraded. However, with the emergence of CDIs and P2P networks, the
network provisioning has become more complicated, since the network loads tend
to shift depending on the content that is currently transported while the
direct peering might not be effective anymore.

\paragraph{Volatile Content Traffic.} CDIs and P2P networks strive to optimize
their own operational overhead, possibly at the expense of the underlying
infrastructure. In terms of CDIs, this means that a CDI chooses the best server
based on its own criteria, not knowing what parts of the networks
infrastructure is being used. Especially with globally deployed CDIs it becomes
increasingly difficult for ISPs to predict what CDI is causing what traffic
from where based on past behavior. This has a direct implication on the traffic
engineering of the network, as this is usually based on traffic predictions
from past network traffic patterns.

\paragraph{Customer Satisfaction.} Regardless of the increased difficulty with
network provisioning and traffic engineering, end-users are demanding more and
larger content. This, coupled with the dominant form of flat rates for customer
subscriptions, increases the pressure on ISPs to delay network upgrades as long
as possible to keep prices competitive. But letting links run full increases
packet loss. This, in turn, drastically reduces the quality of experience of
the end-users. This, in turn, encourages end-users to switch their
subscriptions.


\subsection{Summary}

In summary, we identify the following challenges in todays content delivery:

\begin{itemize*}

\item The ISP has limited ability to manage its traffic and therefore incurs
  potentially increased costs, e.g., for its interdomain traffic, as well as
  for its inability to do traffic engineering on its internal network while
  having to offer competitive subscriptions to its end-users.

\item The P2P system has limited ability to pick an optimal overlay topology
  and therefore provide optimal performance to its users, as it has no prior
  knowledge of the underlying Internet topology. It therefore has to either
  disregard or reverse engineer it.

\item The CDI has limited ability to pick the optimal server and therefore
  provide optimal performance to its users, as it has to infer the network
  topology as well as the dynamic network conditions. Moreover, it has limited
  knowledge about the location of the user as it only knows the IP address of
  the DNS resolver.

\item The different systems try to measure the path performance independently.

\end{itemize*}
