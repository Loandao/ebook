%%%%%%%%%%%%%%%%%%%%%%%%%%%%%%%%%%%%%%%%%%%%%%%%%%%%%%%%%%%%%%%%%%
\section{Opportunities for Collaboration}\label{sec:Opportunities}
%%%%%%%%%%%%%%%%%%%%%%%%%%%%%%%%%%%%%%%%%%%%%%%%%%%%%%%%%%%%%%%%%%

As pointed out ISPs are in a unique position to help CDIs and P2P systems to
improve content delivery since they have the knowledge about the state of the
underlying network topology, the status of individual links, as well as the
location of the user. In this section we first describe the high level concept
all existing solutions have in common and then continue by illustrating where
and why they differ in certain aspects.

The presented solutions include the original Oracle concept proposed by
Aggarwal et al~\cite{afs-cispp2pcip-ccr07}, P4P proposed by Xie et al.~\cite{p4p},
Ono proposed by Choffnes and Bustamante~\cite{taming} and PaDIS proposed by Poese
et al.~\cite{PaDIS-IC}. We also give an overview of the activities within the
IETF which have been fueled to some extend by the proposed systems discussed in
this section, namely ALTO and CDNi.

\subsection{Conceptual Design}\label{sec:Concept-Opportunties-Collaboration}

To overcome the challenges in Content Delivery, recall
section~\ref{sec:challenges}, various solutions have been proposed by the
research community. While they all differ in certain aspects, their basic idea
is the same: utilize available information about the network to make an
educated selection prior connecting to a service.  Following this idea, all of
the proposed solution employ the same basic conceptual design: the
\emph{management plane} is responsible for collecting up-to-date information
about the network while the \emph{control plane} acts as an interface to this
information for the application.\\

\noindent {\bf Management Plane: The Network Map.}
The systems management plane is responsible to collect up-to-date state network
information, such as network topology, routing information, link utilization
and other important metrics. This information is used to maintain an internal
map of the network representing the current state of the real network.  One
important aspect of this component is how the information about the network is
retrieved. The different implementations range from active measurements over
passive measurements to active participation in network management systems
(such as BGP). Another important aspect is the frequency in which the
information is collected. For certain information such as topology or routing
an immediate update is necessary to guarantee correct functioning of the
system, while others, such as link utilization or packet loss rates, only
degrade the quality of the system. Still other information, such as link
capacities or transmission delays, can be considered (semi-)static.  Last but
not least the systems differ in what information is necessary to be
operational and if additional information sources can be used to improve
accuracy.\\

\noindent {\bf Control Plane: The Information Interface.}
The control plane of the system is responsible for providing an interface to
the information of the management plane so that clients can make use of the
information. This can basically be seen as an interface or API that clients can
query to get information about the current network state. The various proposed
solutions differ mainly in which fashion and at which granularity the
information can be retrieved. There are two main competing approaches:
abstracted network maps and preference lists. The first one transforms the
available information from the management plane into an annotated
representation of nodes and edges. The big difference to the actual data of the
management plane is the aggregation level and the specific annotations. Clients
can then query the system to get an up-to-date abstract network map, which they
can use to decide which of the possible destination to connect to by
calculating the best candidates by themselves using their own optimization
target. The second one uses the information of the management plane to create a
ranked list of possible service destinations (read: IP addresses). The required
input includes the source, possible destinations and (if the system supports
multiple collaboration objectives) an optimization goal, \eg minimal delay. The
output consists of a re-ordered list of the possible destinations in regard to
the optimization goal, the first being the most and the last being the least
desirable destination.

Note that in both cases the client is in the position to select the final
destination, allowing to completely ignore the additional information. Another
important fact is that the client is not necessarily the end-user but might be
a service provider themselves. For instance a company providing content
delivery service (CDN) could make use of this service to improve its
user-to-server mapping accuracy or in case of the BitTorrent P2P system the
tracker could query the service prior returning an initial peer list to a
connected client.  While not strictly necessary, the two components are
usually implemented as separate entities within the system to allow better
scalability, information aggregation and/or anonymization without loosing
precision or multiple collaboration objectives. In addition to that, all
systems table important issues for any collaboration approach, such as privacy
information leakage or targeted objective(s).

The presentation of the following solutions will outline the specific
implementation and thus highlights the differences between the solutions.

\subsection{P2P Oracle Service}\label{sec:p2p-collaboration}

Aggarwal et al.~\cite{afs-cispp2pcip-ccr07} describe an \emph{oracle} service to
solve the mismatch between the overlay network and underlay routing network in
P2P content delivery. Instead of the P2P node choosing neighbors independently,
the ISP can offer a service, the \emph{oracle}, that ranks the potential
neighbors according to certain metrics: a client supplied peer list is
re-ordered based on coarse-grained distance metrics, \eg the number of AS
hops~\cite{routing-book-00}, the peer being inside/outside the AS or the
distance to the edge of the AS. This ranking can be seen as the ISP expressing
preference for certain P2P neighbors.  For peers inside the network additional
information can be used, such as access bandwidth, expected delay or link
congestion to further improve the traffic management.


\subsection{Proactive Network Provider Participation for P2P (P4P)}\label{sec:P4P}

The ``Proactive Network Provider Participation for P2P'' is another approach to
enable cooperative network information sharing between the network provider and
applications. The P4P architecture~\cite{p4p} introduces iTrackers as portals
operated by network providers that divides the traffic control responsibilities
between providers and applications. Each iTracker maintains an internal
representation of the network in the form of nodes and annotated edges.  A node
represents a set of clients that can be aggregated at different levels, \eg
certain locations (PoP) or network state (similar level of congestion).
Clients can query the iTracker to obtain the ``virtual'' cost for possible peer
candidates. This ``virtual'' cost allows the network operators to express any
kind of preferences and may be based on the provider's choice of metrics,
including utilization, transit costs, or geography. It also enables the client
to compare and choose the most suited peers to connect to.


\subsection{Ono - Travelocity-based Path Selection}\label{sec:Ono}

The Ono system~\cite{taming} by Choffnes and Bustamante is based on
``techniques for inferring and exploiting the network measurements performed by
CDNs for the purpose of locating and utilizing quality Internet paths without
performing extensive path probing or monitoring'' proposed by Su \etal
in~\cite{DraftingAkamai:SIGCOMM2006}. Based on their observations that CDN
redirection is driven primarily by latency~\cite{DraftingAkamai:SIGCOMM2006},
they formulate the following hypothesis: Peers that exhibit similar redirection
behavior of the same CDN are most likely close to each other, probably even in
the same AS.  For this each peer performs periodic DNS lookups on popular CDN
names and calculates how close other peers are by determining the cosine
similarity with their lookups. To share the lookup among the peers they use
either direct communication between Ono enabled peers or via distributed
storage solutions \eg DHT-based.  On the downside Ono relies on the precision
of the measurements that the CDNs perform and that their assignment strategy is
actually based mainly on delay. Should the CDNs change their strategy in that
regard Ono might yield wrong input for the biased peer selection the authors
envision.

When considering our design concept described above, Ono is a bit harder to fit
into the picture: Ono distributes the functionality of the management and
control planes among all participating peers.  Also, Ono does not try to
measure the network state directly, but infers it by observing Akamai's
user-to-server mapping behavior on a large scale and relies on Akamai doing the
actual measurements~\cite{Akamai-Network}, Thus the management plane of Ono
consists of recently resolved hostnames from many P2P clients. The quality of
other peers can then be assessed by the number of hostnames that resolve to the
same destination.  The control plane in Ono's case is a DHT, which allows
decentralized reads and writes of key-value pairs in a distributed manner, thus
giving access to the data of the management plane.

\subsection{Provider-aided Distance Information System (PaDIS)}\label{PaDIS-Opportunities-for-Collaboration}
\newcommand{\assignmentenabler}{{informed user-server as\-sign\-ment}\xspace}
\newcommand{\assignment}{{user-server as\-sign\-ment}\xspace}
\newcommand{\assignmentproblem}{{informed user-server as\-sign\-ment problem}\xspace}


In~\cite{PADIS2010,PaDIS-IC} Poese et al. propose a ``Provider-aided Information Systems
(PaDIS)'', a system to enable collaboration between network operators and
content delivery systems. The system enhances concept of the P2P Oracle to
include server based content delivery systems (\eg CDNs), to maintain an
up-to-date annotated map of the ISP network and its properties as well as the
state of ISP-operated servers that are open for rent. In addition, it provides
recommendations on possible locations for servers to better satisfy the demand
by the CDN and ISP traffic engineering goals.  In the management plane, it
gathers detailed information about the network topology, \ie routers and links,
annotations such as link utilization, router load as well as topological
changes. An Interior Gateway Protocol (IGP) listener provides up-to-date
information about routers and links.  Additional information, \eg link
utilization and other metrics can be retrieved via SNMP. A Border Gateway
Protocol (BGP) listener collects routing information to calculate the paths
that traffic takes through the network, including egress traffic. Ingress
points of traffic can be found by utilizing Netflow data. This allows for
complete forward and reverse path mapping inside the ISP and enables a complete
path map between any two points in the ISP network.  While PaDIS builds an
anotated map of the ISP network, it keeps the information acquired from other
components in separate data structures. This separation ensures that changes in
prefix assignments do not directly affect the routing in the annotated network
map.  Pre-calculating path properties for all paths, allow for constant lookup
speed independent of path length and network topology. On the control plane,
PaDIS makes use of the prefrence lists known from the P2P Oracle, but supports
multiple, individual optimization targets. Apart from basic default
optimizations (\eg low delay, high throughput), additional optimizations can be
negotiated between the network operator and the content delivery system.
For CDN-ISP collaboration opportunities when the ISP operates both the network and the CDN we
refer the reader to~\cite{CooperativeISPCDN,TECDN,N-CDN,CTTE2010}.

\subsection{Application-Layer Traffic Optimization (ALTO)}\label{sec:ALTO}

The research into P2P traffic localization has led the IETF to form a working
group for ``Application Layer Traffic Optimization (ALTO)''~\cite{ALTO-WG}.
The goal of the ALTO WG is to develop Internet standards that offer
``better-than-random'' peer selection by providing information about the
underlying network and to design a query-response protocol that the
applications can query for an optimized peer selection
strategy~\cite{ietf-alto-protocol}.  On the control plane, ALTO offers multiple
services to the Applications querying it, most notably are the Endpoint Cost
Service and the Map service. The Endpoint Cost Service allows the Application
the query the ALTO server for costs and rankings based on endpoints (usually IP
subnets) and use that information for an optimized peer selection process or to
pick the most suitable server of a CDI. The Network Map service makes use of
the fact that most endpoints are in fact rather close to each other and thus
can be aggregated into a single entity. The resulting set of entities is then
called an ALTO Network Map. The definition of proximity in that case depends on
the aggregation level, in one Map endpoints in the same IP subnet may be
considered close while in another all subnets attached to the same Point of
Presence (PoP) are close. In contrast to the Endpoint Cost Service the ALTO
Network Map is suitable when more Endpoints need to be considered and offers
better scalability, especially when coupled with caching techniques. Although
the ALTO WG statement is more P2P centric, the service is also suitable to
improve the connection to CDN servers.

