%\ed{A one page conclusion.} 
%Something on evolution of transport
%Something on Middleboxes
%Something on MPTCP
%Something on Minion
%One paragraph on the future

The Transport Layer in the Internet evolved for nearly two decades,
but it has been stuck for over a decade now.
A proliferation of middleboxes in the Internet,
devices in the network that look past the IP header,
has shifted the waist of the Internet hourglass 
upward from IP
to include UDP and TCP,
the legacy workhorses of the Internet.
While popular for many
different reasons, 
middleboxes thus deviate from the Internet’s
end-to-end design, creating large deployment
“black-holes”---singularities where legacy transports get through, but
any new transport technology or protocol fails,
severely limiting transport protocol evolution.
The fallout of this ossification
is that new transport protocols,
such as SCTP and DCCP,
that were developed to offer much needed 
richer end-to-end services to applications,
have had trouble getting deployed since they require changes 
to extant middleboxes.

Multipath TCP is perhaps the most significant change to TCP in the past twenty
years. It allows existing TCP applications to achieve better
performance and robustness over today's networks, and it has been
standardized at the IETF. The Linux kernel implementation shows that
these benefits can be obtained in practice.  However, as with any
change to TCP, the deployment bar for Multipath TCP is very high: only
time will tell whether the benefits it brings will outweigh the added
complexity it brings in the end-host stacks.

The design of Multipath TCP has been a lengthy, painful process that took around five
years. Most of the difficulty came from the need to support existing middlebox
behaviors, while offering the exact same service to applications as TCP. Although
the design space seemed wide open in the beginning, in the end we were \emph{just} able
to evolve TCP this way: for many of the design choices there was only one viable option
that could be used. When the next major TCP extension is designed in a network with even more
middleboxes, will we, as a community, be as lucky?

A pragmatic answer to the inability to deploy new transport protocols
is Minion. It allows deploying new transport services 
by being backward compatible with middleboxes by encapsulating new protocols inside TCP.
Minion demonstrates that it is possible to obtain
unordered delivery and multistreaming
from wire-compatible TCP and TLS streams
with surprisingly small changes to TCP stacks
and application-level code.
Minion offers a path toward
the performance benefits of unordered delivery,
which we expect to be useful to applications
that use TCP for a variety of pragmatic reasons.

%Minion is just the latest in an ever-growing encapsulation war that 
%has engulfed end-hosts and network providers for years on end. 
Early in the Internet's history,
all IP packets could travel freely through the Internet, as IP was the narrow waist of
the protocol stack. 
Eventually, 
apps started using UDP and TCP exclusively,
and some, such as Skype, used them adaptively,
probably due to security concerns
in addition to the increasing proliferation of middleboxes that allowed only UDP and TCP through.
As a result, UDP and TCP over IP were then perceived to constitute the new waist of the Internet.
(We'll note that HTTP has also recently
been suggested as the new waist~\cite{popa10http}.) 
%UDP is not that great getting through newer firewalls either, so TCP is the new narrow
%waist, or even HTTP as recently proposed. 

Our observation is that
whatever the new waist is,
middleboxes will embrace it and optimize for it: 
if MPTCP and/or Minion become popular, 
it is likely that middleboxes will be devised that understand these protocols
to optimize for the most successful use-case of these protocols
and to help protect any vulnerable applications using them.
One immediate answer from an application would be to
use the encrypted communication proposed in Minion---%
but actively hiding information from a network operator
can potentially encourage the network operator
to embed middleboxes that intercept encrypted connections,
effectively mounting man-in-the-middle attacks
to control traffic over their network,
as is already being done in several corporate firewalls~\cite{marko10using}.
%a potential serious security breach, and many ISPs
%may either drop traffic or require their customers to ``install'' keys in the 
%browser that effectively allow middleboxes to become a man-in-the-middle of the TLS connection. 
To bypass these middleboxes,
new applications may encapsulate their data {\em even} deeper,
leading to a vicious circle resembling an ``arms race'' for control
over network use.
% them looking even deeper in the packets. As we can see, 
%this is a vicious circle that no-one can ever win.

This ``arms race'' is a symptom of a fundamental tussle between
end-hosts and the network: end-hosts will always want to deploy new
applications and services, while the network will always want to allow and optimize only
existing ones~\cite{tussle}.
To break out of this vicious circle,
we propose that
end-hosts and the network must co-operate,
and that they must build cooperation into their protocols.
Designing and providing protocols and incentives for this cooperation may hold the key to creating a
truly evolvable transport (and Internet) architecture.
