%\clearpage
\section{Conclusion}
\label{sec:conclusion}

This chapter has aimed at clarifying the state of the art in Internet
topology measurement and modelling, and correcting a number of clear
and present flaws in reasoning. As we outlined in the introduction, we
can see a number of themes recurring at multiple levels of hierarchy
in topology modelling: 

\begin{description}

\item[{\bf Theme 1:}] When studying highly-engineered systems such as
  the Internet, ``details'' in the form of protocols, architecture,
  functionality, and purpose matter.

\item[{\bf Theme 2:}] When analyzing Internet measurements, examining
  the ``hygiene'' of the available measurements (\ie an in-depth
  recounting of the potential pitfalls associated with producing the
  measurements in question) is critical.

\item[{\bf Theme 3:}] When validating proposed topology models, it is
  necessary to treat network modeling as an exercise in
  reverse-engineering and not as an exercise in model-fitting.

\item[{\bf Theme 4:}] When modeling highly-engineered systems such as
  the Internet, beware of M.L. Mencken's quote ``For every complex
  problem there is an answer that is clear, simple, and wrong.''

\end{description}

We have not tried to survey the entire literature in this area, and we
apologize to those whose work has not appeared here, but there are
other extant surveys mentioned at the relevant points throughout this
chapter, for specific components of the work. We also have not tried
to critique every model, but rather tried to provide general guidance
about modelling. It is intended that the readers could themselves
critique existing and new models based on the ideas presented here.

In addition, we do not claim to have covered every type of topology
associated with the Internet. Specifically, we have avoided topologies
at the applications layer, for instance those associated with the WWW
or online social networks.  We made this choice simply because these
topologies are (despite being ``Internet'' topologies) profoundly
different from the topologies we have included. They are almost purely
virtual whereas all of the networks considered here have a physical
component, which leads to the arguments for optimization as their
underlying construction.  An important open problem in this context is
the role that societal-related factors play over more economic- or
technology-based drivers in the formation and evolution of these
virtual topologies.

Finally, in each section, we have aimed at illuminating some of the
current problems and identifying hopefully fruitful directions for
future research in this area.
